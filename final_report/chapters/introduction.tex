%!TEX root = ../thesis.tex

\chapter{Introduction}
\label{chap:i}

\section{Motivation}
\label{sec:motivation}

Constructing and maintaining an up-to-date graph-like road network on the national level has a range of firmly established uses. Owing to its structure, it can be used efficiently for modelling and simulation purposes, such as traffic flow simulations, passenger transport modelling, construction and upgrade impact modelling (to pinpoint optimal locations and types of investment), and traffic noise load modelling (\cite{bell_lida_1997, zhu_li_2007, zhang_2011, duran_santos_2014, peng_etal_2020}). It can also be used for navigation; a graph-like road network representation is at the heart of most road navigation services (\cite{yue_etal_2008}). Combined with other data sets, we can mention an even wider range of use cases: complemented by ecological statistics and models, it can offer insight into the impact of the presence of roads, and planned road construction on the flora and fauna in their vicinity.

Or to mention a different type of example, an accredited model representing the road network in a digital format may be used as a shared working space when aggregating geospatial data relating to road infrastructure from various sources. It makes it possible for geographical road locations, topographical relationships, and arbitrary semantic information to reside in the same network-type data model, making analysis techniques more straightforward, enforcing consistency and saving effort for data providers who would otherwise all need to maintain their own road model (\cite{ekpenyong_etal_2007}). This example is closely related to the ambitions the provider of the Dutch digital road network, which is the primary subject of this research, has with their road network model.

One may remark that a two-dimensional representation with \textit{approximate} geographical locations may suffice for many of the purposes I listed as examples above, topology being the main concern in network analysis. For instance, \ac{gnss} navigation software often use snapping methods to ensure that the navigating vehicle always traverses the road graph – ensuring that even for imperfect positioning results navigation remains continuous (\cite{fouque_bonnifait_2008, chen_hsu_2020}). Traffic flow simulations are primarily concerned with traffic loads, road properties, and how roads are subdivided by intersections among other aspects. \textit{Mostly}, they are not concerned with the exact geographical locations of roads – as long as the topology is relatively accurate, any geographical permutation of the network will yield largely invariant results (\cite{thomson_richardson_1995}).

However, some applications are concerned with the road network in the context of its surroundings, which makes the accuracy of its georeferencing a relevant topic. Noise modelling is such an application, because modelling the propagation of road noise concerns deriving the noise load affecting various objects in the vicinity of the the roads. Furthermore, the propagation of the noise itself requires taking into account objects that may suppress or amplify the noise, such as buildings, noise barriers and terrain (\cite{ishiyama_etal_1991, bennett_1997, guarnaccia_quartieri_2012}). A realistic noise propagation model takes into account the height of the road, as well as the vertical extent of surrounding objects. However, typical digital representations of national road networks tend to only describe lateral road positions, a factor that may limit their use in such noise propagation applications. For instance, consider two buildings at the side of a road, a tall one \textit{behind} a shorter one. A 2D semantic road model would tell us that the entirety of the taller building receives a decreased noise load, because the shorter one – positioned between the taller building and the road – suppresses it. In reality, this observation is invalid because the part of the taller building which is visible from the road above the shorter building receives the full noise load. To be able to represent such 3D relationships in our model correctly, we need both buildings and roads to be fully three-dimensional.

2D-projected digital road models with mediocre accuracy have attracted great scientific and commercial attention since the advent of digital cartography and satellite navigation (\cite{taylor_etal_2001, fouque_bonnifait_2008, yue_etal_2008, chen_hsu_2020}). However, \textit{accurate 3D representations} are still atypical, owing to factors such as increased cost of generation and maintenance, increased complexity of visualisation and analysis, and a lack of significant use cases (\cite{zhu_li_2007, wang_etal_2014}). As a result, 2D road models are common in terms of both public and private geospatial providers, whereas accurate 3D road models are rare in comparison.

When a use case arises and an accurate 3D model is needed, providers generally face two choices: to produce a new model, or to enrich an existing 2D model with elevation data. The decision generally depends on the quality of the available 2D data set relative to the requirements for the 3D model, as well as that of the dataset(s) available as sources of elevation data, with which the 2D model can be enriched, among other factors (\cite{zhu_li_2007, zhu_li_2008, wang_etal_2014}). In terms of the source of elevation data, the rule of thumb in the geospatial field is that data acquisition is far more expensive than re-using existing datasets, especially openly available ones. As a result, many providers first attempt to find a way to convert their datasets into 3D using existing data in such a cost-effective manner.

\section{The NDW commission}
\label{sec:commission}

In certain projects the accuracy requirement and restrictions on the modelling procedure may be prescribed legally. Such is the case for the client of the present dissertation research, the \ac{ndw} (National Road Traffic Data Portal), part of \ac{rws} (Directorate-General for Public Works and Water Management), a Dutch government organisation who are in the process of enriching their pre-existing open data 2D road model, called NWB (National Road Database), with 3D data, to attain compliance with the new version of the Dutch noise legislation or \textit{geluidwetgeving}, coming into effect on the 1\textsuperscript{st} of January 2022. The new version of the legislation prescribes, among other things, a horizontal \textit{and vertical} accuracy of 20 centimetres for the road model underlying the noise simulations. Due to cost considerations and reasons related to \ac{ndw}’s data acquisition pipeline, the pre-existing 2D version of NWB will be converted into a 3D dataset (dubbed \textit{3D-NWB}) primarily using open data geospatial datasets. They have produced a prototype implementation themselves, and subsequently contracted the consultant firm \ac{rhdhv} to create a commercial implementation based on their experience with the prototype. The development of this tool was concluded in December 2020, with a preliminary version of the results already publicly available on their website in addition to the traditional 2D version.

Thus for \ac{ndw}, the next year will be about assessing the quality of their new product and improving it as they see necessary. In particular, they wish to assess how it fares in terms of the requirements set by the law. This dissertation research attempts to contribute to this assessment by presenting an original system design and implementation that favours scientific correctness, and in which output accuracy can be qualitatively and quantitatively examined. By comparing the results of this academic implementation to that of the commercial one, it becomes possible to indirectly evaluate the commercial results' general quality and accuracy in an indirect manner. My work also explores various related geomatics topics in the process, which I specifically refer to as the academic aims of this project, in the present report.

My research was carried out in consultation with the above commercial parties. In fact, the stages leading up to the submission of my dissertation proposal involved continuous consultation with personnel both at \ac{ndw} and \ac{rhdhv} while the commercial implementation was still being developed, to ensure that my research fits well with \ac{ndw}'s plans and the commercial implementation - in addition to answering various questions inspired purely by academic interest. In simple terms, the initial idea for this project was to complete \ac{ndw}'s commission - same as the commercial project - but by designing a system that builds primarily on geomatics tools and the results of which can be analysed in terms of empirical or formal mathematical accuracy.

\section{Field and relevance}
\label{sec:relevance}

For reasons that will later become clear (see Section \ref{sec:input}), I primarily focused on a Lidar point cloud and a 3D topographical line dataset as elevation sources. In both of these datasets, it is clearly evidenced that roads are occasionally in complex three-dimensional relationships with one another and with their environment. For instance, they cross above and below other roads and are also frequently occluded by other objects such as vegetation and buildings. Already in the planning phase of the project the question had arisen, how such real-world geometries should be dealt with in the conversion process - evidently they will require special treatment relative to well-exposed road surfaces. The answer to this question is closely linked with which field of geoscience my project is positioned in.

The likely candidates in the context of digital road network modelling are \ac{gis} and geomatics. It is thus worth discussing briefly how each typically treats 3D objects. One of the reasons why 2D road models are popular is that their geometry and network properties can be analysed using a multitude of well-proven \ac{gis} methods and software kits. However, in \ac{gis} models, even if elevation measurements exist, they are generally only present as an \textit{elevation attribute} (i.e. a semantic data field, like street names), because \ac{gis} geometrical models do not typically support true 3D operations. This is conceptually identical to projecting the geometries onto the horizontal plane. Geometric models that treat the vertical dimension explicitly are more common in geomatics; namely 2.5D and 3D models. While using 2.5D models restricts the types of physical entities that can be modelled, it also greatly simplifies certain types of analysis conceptually and computationally. This makes it ideal for working on similar scales to \ac{gis}; on the national scale for instance, as in this research.

While 2.5D modelling initially appears to be a good candidate for this project, we may observe that it is by definition unsuitable for handling the 3D relationships that roads have with each other and with their environment. However, much like how the concept of divide-and-conquer works in computer science, it is also possible decompose three-dimensional, geometrical problems into smaller sub-problems until they become natively compatible with 2.5D methods which are simpler to solve individually than the 3D problem as a whole. This research is positioned in the field of geomatics because my system design is specifically intended to explore how 2.5D methods can be applied in a way that enables the \textit{piecewise} modelling of a national road network. The divide-and-conqer concept will be applied to decompose the road network into segments that can be individually, locally regarded as \textit{terrain} (i.e. a mathematical surface) and hence be modelled in 2.5D.

Geomatics is comprised of a wide range of disciplines, several of which are relevant to the present research. As it focuses on 2.5D methods to a great extent, it overlaps with the geomatics field of \textit{digital terrain modelling} in terms of how it generates and stores the digital representation of road surfaces, and as a consequence, the manner in which it will derive elevations from them: using \textit{spatial interpolation}. As the overview of the methods in Section \ref{sec:methodsoverview} reveals, it also strongly overlaps with the geomatics discipline of \textit{feature extraction} (and to a lesser extent, \textit{photogrammetry}), because of the intermediate steps used by the pipeline to derive the 2.5D road surface models from our input datasets.

This is in line with my goal to study how a combination of mainly geomatics-based tools and methods can be used to accomplish the tasks required by \ac{ndw}, and to also assess their accuracy and suitability when used in this way. However, I also use \ac{gis} methods "under the hood" in all parts of the project - for instance, 2D geometry intersection tests, orientation tests and spatial queries are pervasive in the implementation, and are thus often mentioned in the detailed description of the processing steps found in Section \ref{sec:methods}. Furthermore, my research often touches on mathematics and statistics, for instance I use polynomial fitting and maximum likelihood estimation (\ac{mle}) throughout the implementation (as examples of the former) and metrics such as standard deviation and RMSE (as examples of the latter).

\section{Research questions}
\label{sec:rq}

My main research question is \textit{"How can we achieve a 3D conversion of the NWB dataset using Dutch open geospatial data and a primarily 2.5D-based surface modelling methodology, while guaranteeing optimal and quantifiable accuracy and completeness?"}. It was distilled from the main areas of interest that we settled on during the preparatory stages of the project, while planning the project with my academic mentors, as well as \ac{ndw} and \ac{rhdhv}. The question is comprised of two halves, which I initially intended to devote equal amounts of attention to - the question of devising a system design and implementing it, and that of assessing its effectiveness and the accuracy and completeness of the output it generates (as well as comparing it with the commercial results).

During the realisation of the project, I eventually settled on focusing more time and effort on the system design and integration, both because it required more time than initially anticipated, and because the quality and accuracy assessment of the results (and their comparison with the commercial results) was more straightforward than expected, leaving more time to work on the implementation. Furthermore, I found that many of the accuracy-related questions depend strongly on the exact specifications of the system design and its implementation, meaning that not all the work necessary to answer those questions was clearly separated into a dedicated accuracy assessment process - much of it needed to be considered and evaluated during the development process.

The two halves of the main research question were created by collecting my sub-questions into these two categories; pipeline design and implementation, and accuracy assessment. Below I present some of these sub-questions, to characterise in somewhat more detail, what specifically my attention was directed towards in this research project. 

\begin{enumerate}
    \item Sub-questions related to \textit{performing} the elevation-enrichment of NWB using Dutch open geospatial data and predominantly via 2.5D geomatics methods
    \begin{enumerate}
        \item What are the exact methods of the commercial implementation and what do we suspect its theoretical shortcomings to be?
        \item Does the literature suggest any methods that are particularly suitable to this research? If so, can we make use of them in our own methods?
        \item How can the academic methods best make use of the combined information content of the datasets that the commercial implementation uses?
        \item How should the road network be subdivided into parts that each represent a 2.5D problem? Can they be processed individually to facilitate easy parallel processing?
        \item As we are using Lidar data, can we produce an accurate and complete \ac{tin} surface model for each 2.5D "unit"? Can we interpolate elevations for NWB through this model?
        \item How do we "stitch" the results of the individual 2.5D procedures back together into a 3D road network with the correct topology?
        \item Can the implementation be made robust enough to handle all (or \textit{most}) challenging road layouts correctly, such as complex motorway junctions?
        \item How can we make the implementation perform well in areas where input elevation data is scarce or missing over longer distances, such as in tunnels?
        \item Can the computational complexity of the program be kept low enough to be suitable for processing all the relevant roads?
        \item While solving \ac{ndw}'s specific problem, can we also ensure that our solution generalises well to other problems of a similar type?
        \item Can the pipeline and the implementation be clearly separated into "modules" to facilitate easy partial reuse?
    \end{enumerate}
    \item Sub-questions related to the \textit{assessment} of the overall quality, completeness and accuracy of the output and its similarity to the commercial results
    \begin{enumerate}
        \item In related work, what methods are typically used to measure empirical and theoretical output accuracy?
        \item According to related work, what typically defines output accuracy? Do local factors also play a role, or is it reasonable to estimate it for the procedure globally?
        \item What is the accuracy of our input elevation data sources? Can we structure the pipeline in a way that their input accuracy can be propagated to the output in a straightforward manner?
        \item To facilitate the above, can we derive our output directly from input data points, despite the large number of processing steps that are potentially necessary?
        \item What is the effect of uncertainty in the lateral positions of NWB centrelines on the effectiveness of our methods, and on the output accuracy?
        \item The road surface model \ac{tin}s are also important products of the pipeline. How can we assess the overall quality and completeness of these?
        \item Can we indicate it in the output, which input elevation source each output elevation estimate was derived from, and to use this capability to derive the output accuracy from the appropriate input dataset's accuracy?
        \item How are temporal inconsistencies between the datasets manifested in the output? Can these be detected by the processing steps?
        \item In case we can estimate accuracy on the local scale, what physical features or sensing issues do drops in accuracy correspond to? If this corresponds to problems with the input, what properties of the input should be improved?
        \item How good is the agreement between the commercial and the academic results? What physical features or sensing issues do disagreement between the results correspond to?
        \item Under what conditions does the academic solution perform better or worse than the commercial one, and what does this tell us about the two underlying sets of methods?
    \end{enumerate}
\end{enumerate}