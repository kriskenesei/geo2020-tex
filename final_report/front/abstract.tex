\chapter*{Abstract}

This research concerned the design and implementation of a system to perform the 3D conversion of a national road network using airborne Lidar data and land-based road elevation measurements, preserving its topology and quantifying output elevation accuracy. While the system was designed with generality in mind, specific Dutch open data datasets were used as input which were also used in a commercial project with the same goals. In addition to designing a scientifically sound equivalent of the commercial implementation, this research also explored purely academic goals, such as selecting road surface laser reflections accurately and using them for 2.5D road surface modelling purposes. Our system was based on decomposing the 3D conversion task into individually solvable 2.5D sub-problems to enable the use of efficient terrain modelling methods, and to prove that scaling via multiprocessing would be possible. At the 10-30 points per m\textsuperscript{2} sampling density we worked with, the road surfaces are oversampled. We found that because our elevation estimates are based on road reflections only, and because road surfaces are smooth and relatively flat, the output quality and accuracy depends only on the specific technique used to extract the elevations from the Lidar data. Using our methods, the output elevation accuracy represents a 30\% increase relative to the Lidar input data, amounting to 10.6 cm at 95\% certainty with our specific datasets. This is assumed to be violated only where local density drops below 3 points per m\textsuperscript{2}, which occurs where roads are occluded by opaque objects, unless the land-based road surface measurements can provide enough additional data. Elsewhere, more than 98\% of the road network conforms with the minimum point density. The 2.5D-based decomposition was found to be an effective means of simplifying the problem and terrain modelling methods were successfully used in the procedure. More than 90\% of the traffic-occupied road surfaces are included in the resulting surface models, and about 75\% of the total paved surfaces. We extracted the output elevations from them, using linear interpolation where they were unavailable. Our 3D conversion quality and accuracy represents an improvement relative to the commercial results, mostly due to the better prioritising of the input elevation sources, in addition to our solution's capacity to handle complex small to medium-scale occlusion better, including complex relationships between roads such as multi-level motorway junctions.