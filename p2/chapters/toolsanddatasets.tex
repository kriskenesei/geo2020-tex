%!TEX root = thesis_proposal.tex

\chapter{Tools and Datasets}
\label{chap:td}

In terms of datasets, both the commercial implementation and the proposed research rely on the following three sources:

\begin{itemize}
\item Nationaal Wegenbestand (NWB, National Road Database)
\item Actueel Hoogtebestand Nederland (AHN, Current Dutch Elevation)
\item Digitaal Topografisch Bestand (DTB, Digital Topographic Database)
\end{itemize}

As mentioned in the section about research questions, I am also planning to examine how specific datasets from individual NDW data suppliers (such as municipalities, road authorities, civil engineering agencies) can be integrated in the procedure and how they affect the accuracy. At this stage in the project, the specific datasets for this aspect have not been chosen yet.

\section*{Nationaal Wegenbestand – NWB}

NWB (or more specifically, NWB-Wegen, the NWB roads product) is an ESRI shapefile-based vector dataset comprised of a semantically rich set of 2D MultiLineString objects. They are interconnected in such a way that NWB can be regarded topologically as a graph representation of the Dutch road network. Although the NWB contains all named and numbered roads in The Netherlands, we are only interested in roads that are semantically marked as state-owned or province-owned (category R for Rijk and P for Provincie), because the new noise regulation is only relevant to these types. NWB and the road types it contains are illustrated in Figure \ref{fig:nwb}. In addition to its important topological graph structure, the NWB lines are georeferenced with good accuracy. Although the accurate nature of its georeferencing is specifically mentioned in the documentation of the data, it is not evidenced by a rigorous evaluation. The quality description includes only a single figure, which is 5-metre accuracy at two standard deviations (i.e. 95 percent confidence), but with the confidence being based on the total number of roads, implying that the accuracy evaluation method is performed on a road-to-road basis, each of which may be comprised of multiple MultiLineString objects. It is not described, how the accuracy is evaluated for each road, for instance whether it is based on the accuracy of the vertex locations of the road, an arbitrary sampling along their length, and how it is aggregated. Furthermore, we do not know whether the reference for the accuracy was empirical (surveyed control points) or whether it is purely theoretical and is based on the sources and the methods involved in creating  and updating the dataset. Both the topological and the geographical information content of NWB is assembled from a wide range of providers ranging from large national providers such as RWS and Kadaster, to local providers such as specific road authorities and civil engineering agencies. No clear indication is given in the documentation about the sources used for the compilation of specific NWB road types or the estimation of their accuracies, but it is mentioned that they are inconsistent across the range of road types. Although the above nominal lateral accuracy (and its description) is not sufficient for compliance with the new noise regulations, the 3D conversion of NWB is not concerned with improving it – in fact, it is a requirement of the project not to displace NWB horizontally. Hence, we may say that the intended outcome of the 3D conversion is to devise a method that can produce accurate elevations for NWB assuming it  is up to SWUNG2 standards. To achieve this in practice, lateral refinement is being done in a separate project by developing a workflow to match NWB roads with roads in another Dutch national open data geospatial dataset called Basisregistratie Grootschalige Topografie (BGT). Since BGT is already compliant with the accuracy requirements, correcting NWB geometries based on BGT will foreseeably solve the lateral accuracy problem of NWB. However, at the time of writing of this report, the BGT-based correction has only been carried out and released for municipality-owned roads (category G for Gemeente), hence both the commercial and the scientific 3D-NWB project still rely on NWB data that lacks this correction. Regardless of improving NWB’s lateral accuracy not being within the scope of this project, it is important to be aware of the uncertainty because of its implications when overlaying NWB with other datasets, such as AHN and DTB.

\begin{figure}
    \centering
    \includegraphics[width=\linewidth]{p2/figs/nwb_sample_02.png} 
    \caption{An example render of NWB. The road segments \textit{(wegvakken)} are colour-coded to indicate management \textit{(wegbeheerdersoort)}. Yellow is used for R-roads (state-managed roads), red denotes P-roads (province-managed roads), magenta means G-roads (municipality-managed roads), and W-roads (RWS-managed roads) are not shown in this figure. Blue roads are T-roads, collecting all roads that do not fall into either of the above 4 categories. In this particular case, they are managed by TU Delft. This extract also contains examples illustrating the challenging geometries that this project will be concerned with. Where the P-road Kruithuisweg/Kruithuisplein crosses the G-road Schoemakerstraat and the R-road A13, no topological interactions are present in 2D, and in fact the roads cross over one another in reality. These locations are indicated in the figure by red circles. Furthermore, where the bike path IJsmeestertunnel (a G-road in NWB) crosses the A13, a series of 3 short tunnels are located in reality. This is indicated in the figure by a blue oval. These are two types of road layouts that we expect to have to dedicate special attention to (although in this project, we will only be dealing with R-roads and P-roads). More information regarding these topics can be found in the Methodology section (\ref{chap:m}).}
    \label{fig:nwb}
\end{figure}

\section*{Actueel Hoogtebestand Nederland – AHN}

\begin{figure}
    \centering
    \includegraphics[width=\linewidth]{p2/figs/ahn_sample_01.png} 
    \caption{An example render of AHN3. It shows the original AHN3 point density (I did not apply thinning). The location is the Knoppunt Deil, SW of Geldermalsen. The white colour corresponds to ground points (class 2), while red corresponds to bridges (class 26). All other classes have been removed. This image illustrates various points. First of all, it shows the clearly defined shapes of motorway lanes and ramps. It also illustrates that where these are found on bridges, the bridge is cleanly, and reliably associated with the right class. It also shows that as soon as the structure ends which is supported by non-ground civil engineering structures, the classification changes back to 2 immediately. In other words, road constructed on elevated ground is reliably classified as ground. Lastly, it shows that even for relatively thin, single-lane bridges, occlusion is a serious problem. Data first becomes gradually sparser at the border and becomes extinct directly below the bridge. Furthermore, this render makes it clear that wherever vehicles were encountered, the sampling is worse locally. It does not go entirely extinct however, owing to each location having been scanned multiple times.}
    \label{fig:ahnbridges}
\end{figure}

AHN are airborne Lidar (ALS) survey datasets available as open government geospatial data in The Netherlands. The surveys are commissioned every few years with AHN1 to AHN3 already complete (1996 to 2019), and the first AHN4 surveys nearing release. At this stage of the project, we are interested in AHN3, although it is a planned side-track of the research to examine how freshly released AHN4 data could best be made use of. The dataset has a combined systematic and stochastic elevation error of 15 centimetres at two standard deviations (i.e. 95 percent confidence), and 6 to 10 points per square metre point density on average, corresponding to a 0.41 to 0.32 metre posting distance. The lateral error is 18 centimetres at two standard deviations. Comparing these to the descriptors of datasets mentioned in research examined as part of our literature review (see the Relevant work section), AHN3 can be considered accurate, and to have excellent point density. The AHN3 point cloud has been classified semi-automatically with great accuracy and is released with the classification included. This means that for most purposes, the point cloud needs not be ground filtered manually, and that extracting certain features is made far easier. Of the various classes available, we are interested in class 2 the most, which corresponds to ground points, and class 26 which contains, among other things, bridges. Ground points are primarily interesting to us because they contain the road points. This includes roads that were constructed directly on the terrain, as well as roads constructed on altered terrain, such as those built on elevated ground or in open trenches. It also contains the points that represent the terrain in the vicinity of the roads. Elevated roads and bridges are not considered ground points and are generally represented by gaps in class 2, which can, in most cases, be filled by using data from class 26, as shown in Figure \ref{fig:ahnbridges}. However, it is worth noting that class 26 also contains other types of objects, for instance large motorway signs arching over the road surface (shown in Figure \ref{fig:ahnsigns}), as well as the civil engineering structures of elevated roads and bridges (in addition to the road surfaces on them). AHN3 is released in the form of a point cloud, as well as DSM and DTM rasters. Both DEMs were produced using basic radial IDW interpolation with a fixed parametrisation. The DTM was generated by including only points from class 2 in the interpolation step. They are both available at 0.5-metre and 5-metre resolutions, with the 0.5-metre resolution being relevant to this project in terms of the target accuracy. For context, this raster converts (on average) 1.5 to 2.5 points into a single raster cell. The Lidar tiles are released in the binary LAZ format (compressed LAS, or LASzip), and are generally several gigabytes in size. As each tile only covers an approximate area of 32 square kilometres and the LAZ compression ratio for this dataset is 0.1 on average, one can see that working with this data can be challenging in terms of computer memory. The methodology section will elaborate on this topic further.

\begin{figure}[h]
    \centering
    \includegraphics[width=\linewidth]{p2/figs/ahn_sample_02.png} 
    \caption{This render is from the same location and data as \ref{fig:ahnbridges}. The only difference is that here it is visible that in preparation for testing, all points falling outside of a 150-metre buffer zone around the NWB centrelines representing these motorways have been removed. It illustrates that while class 26 reliably contains well-defined bridges, it also contains other structures. For instance, full-width motorway signs are part of this classification with similar reliability to bridges.}
    \label{fig:ahnsigns}
\end{figure}

\section*{Digitaal Topografisch Bestand – DTB}

\begin{figure}
    \centering
    \includegraphics[width=\linewidth]{p2/figs/ahn_sample_04.png} 
    \caption{This render is from the same location and data as Figures \ref{fig:ahnbridges} and \ref{fig:ahnsigns}. The differences are that ground points are coloured blue, and that NWB centrelines are shown. As they have no elevation, they appear below AHN3 points and are consequently masked out partially wherever AHN3 points are present above them. The AHN3 points were thinned randomly (less than 30 percent remain) so that the centrelines are better visible. This figure illustrates that in general, there is good agreement between the AHN3 point cloud tiles and the NWB centrelines. In all sample locations examined so far, they were reliably on, or very near to, the road surfaces they represent the centrelines of. The most noticeable deviations from the true centrelines of roads occur where they are intensely curved and where motorway ramps merge into motorway lanes. The reason for the prior appears to mostly be the angularity introduced by the finite discretisation of curves in NWB, which can be clearly observed in this image, for instance in the location pointed out by the green arrow. The succession on three bridges in the bottom of the image do not have an associated NWB centreline because they are railway bridges. However, it is worth noting here that the same would be observed for roads not yet registered in NWB (i.e. newly constructed roads).}
    \label{fig:ahnnwb}
\end{figure}

Like NWB, DTB (or more specifically, DTB-Droog, the DTB roads product) is a Dutch open data geospatial dataset in the ESRI shapefile format. For the purposes of this project, we only need DTB street outlines (edges), hence we can also regard this as a dataset comprised exclusively of ´MultiLineString´ objects, making it identical to NWB in its data structure. DTB is managed by RWS and it is concerned only with state-owned roads and roads on state-leased land, which translate almost exclusively to NWB R-roads. DTB is, hence, not a reliable source of information about provincial roads (NWB P-roads), our other road type of interest. The DTB lines that are commonly used as road edges are not, in fact, the edges of the paved surface. They are called \textit{verflijnen}, and are the approximate locations of the painted lines representing the outermost edges of the area open to traffic. This is a lucky correspondence with NWB, as it too, contains the centrelines of the areas of roads that are open to traffic. The second-best option would be to use the category \textit{gelederail constructie}, which generally provide good approximations of the edges of the paved surfaces on which the \textit{verflijnen} were painted. Unfortunately, these correspond to the location of safety rails, and as such they are often not on the road and are also not present everywhere. Like NWB (but unlike AHN), DTB’s acquisition also concerns several organisations performing various types of sensing, which are then semi-automatically assembled into the complete DTB product. The two types of sensing techniques contributing to DTB road outlines are traditional manual land-based surveying methods and car-mounted Lidar (MLS). The accuracy of DTB is not published by RWS or any other organisation contributing to or using the dataset, although it is generally thought to be quite accurate. It is mostly used for purposes relating to road management and civil engineering and apart from road edges, it contains more than 400 other types of semantically identified roadside object types. As most users of DTB work in GIS environments, particular care is taken to separate objects into layers in such a way that inside any one layer, they never overlap. This means that they can be processed using 2D GIS tools easier, reducing the significance of elevation to a mere semantic detail (non-geometric attribute). Whenever objects in any given such layer are found in the same position (or closer than a certain threshold), they are automatically shifted by a maximum distance of 5 centimetres to eliminate the overlap, most commonly for DTB road signs. While descriptions in the documentation can be found suggesting that DTB lines are measured accurately, I must consider the formal accuracy of DTB unknown due to the lack of any numerical figures or other evidence to support this. Furthermore, as I illustrate in \ref{fig:dtbnwb}, a visual comparison with NWB reveals various types of significant inconsistencies and interpretation issues. A visual comparison with AHN3 is also shown in \ref{fig:dtbahn}.

\begin{figure}
    \centering
    \includegraphics[width=\linewidth]{p2/figs/dtb_sample_07.png} 
    \caption{Renders of DTB \textit{verflijnen} overlaid with NWB centrelines. NWB symbology is identical to the one used in \ref{fig:nwb}, blue dashed lines are the DTB lines (which we consider as road edges). The location is the same as in Figures \ref{fig:ahnbridges} to \ref{fig:ahnnwb}. While in general DTB shows good agreement with NWB, DTB appears to have both a range of individual shortcomings, as well as specific problems in terms of its compatibility with NWB. \textbf{Top left:} there is not necessarily only one \textit{verflijnen} line on either side of NWB (green arrow). Furthermore, this category appears to also be used for the lines where vehicles should stop ahead of an intersection or traffic lights (red arrow). \textbf{Top right:} as mentioned in the context of AHN3, NWB motorway ramps often merge with the motorway lanes at unrealistic angles. This also causes disagreement with DTB, often violating the assumption that NWB centrelines should lie between DTB edges. \textbf{Bottom left:} DTB road edges are often missing. Furthermore, NWB centrelines may be very close to DTB road edges, as indicated by the blue arrow. \textbf{Bottom right:} as already mentioned in the context of NWB’s agreement with the AHN3 point cloud, the discretisation and apparent inaccuracy of NWB often results in angular, jagged lines where roads are curved intensely. This often results in NWB getting very close to DTB edges, and even crossing them occasionally. It is worth pointing out that although this is a 2D visualisation, DTB has 3D geometry. For instance, in the top left, the R-road and P-road edges are topographically correct in the sense that they do not intersect in 2D without having shared vertices at the location of the intersection. They properly pass above one another in three dimensions.}
    \label{fig:dtbnwb}
\end{figure}

\begin{figure}
    \centering
    \includegraphics[width=\linewidth]{p2/figs/dtb_sample_10.png} 
    \caption{Renders of DTB line features overlain with AHN3 in 3D. \textbf{Top:} all DTB lines are shown, illustrating that roadside slopes, canals, ditches, safety rails and other objects are all represented in this dataset. Even the full-width motorway signs are represented (either by lines on the road, or on the structure itself, inconsistently). \textbf{Bottom:} the same visualisation, but with only DTB \textit{verflijnen} shown. The lines are generally in good agreement with AHN3 visually, in terms of their elevations. As this figure illustrates, they are also a viable source of information in the context of identifying which class-26 Lidar points lie on the surface of R-roads. Same location and AHN3 data as in previous figures in this section, but with ground points being shown in green.}
    \label{fig:dtbnwb}
\end{figure}

\section*{Testing data}

Due to its volume, it will not be possible to develop and test the implementation without cropping AHN3 Lidar data. Even after the implementation of the segmentation workflow, the size of the testing point clouds will need to be manageable to facilitate efficient debugging. At the same time, they will need to be sufficiently representative to provide an insight into both the computational complexity and the visual performance of the algorithm in various types of environments. I have already selected and pre-processed several AHN3 tiles into medium-sized point clouds that will be further cropped when developing and testing specific parts of the implementation. The table below provides an inventory of the testing datasets I produced by selecting various lengths of NWB centrelines at interesting road features and keeping only those Lidar points which are within a 150-metre buffer distance from the selected road centrelines. The resulting datasets fit into memory easily even when extracted to LAS. The tile numbers in the first column are from: https://downloads.pdok.nl/ahn3-downloadpage/

\textit{[Inventory of testing datasets...]}

\section*{Tools}

In terms of tools, for visualisation purposes I will use QGIS and CloudCompare, with potential additions in later stages of the project. In terms of coding, the source code will be written in Python 3, with a strong dependence on pre-existing packages. The final set of tools will be determined when implementing the methodology. To give a few examples, I may use native modules such as fiona and shapely for handling basic vector operations, and PDAL and CGAL bindings as well as binaries such as GDAL and LASTools for point cloud processing and various geometric operations such as terrain modelling. GitHub will be used for source code management.