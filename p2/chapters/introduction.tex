%!TEX root = ../thesis_proposal.tex

\chapter{Introduction}
\label{chap:i}

\textit{Section being written}

Constructing and maintaining an up-to-date graph-like road network on the national level has a range of firmly established uses. Owing to its structure, it can be used efficiently for modelling and simulation purposes, such as for traffic flow simulations, construction and re-construction impact modelling (to pinpoint optimal locations and types of investment), and noise propagation modelling (where the roads are a source of noise load). It can also be used for navigation; a graph-like road network representation is at the heart of all road navigation services. Combined with other data sets, we can mention an even wider range of use cases: complemented by ecological statistics and models, it can offer insight into the impact of roads, and future road construction on the ecological impact on the flora and fauna in the vicinity of the roads. Or to generalise, a good model representing the road network in a digital format may be used as a shared working space when aggregating geospatial data relating to roads from various sources. It makes it possible for geographical road locations, topographical relationships, and arbitrary semantic information to reside in the same data model, making analysis techniques more straightforward, enforcing consistency and saving effort for data providers who would otherwise all need to maintain their own road model.

One may remark that a two-dimensional representation with approximate geographical locations may suffice for many of these purposes, topology being the main concern in network analysis. For instance, GNSS navigation software often use snapping methods to ensure that the navigating vehicle always traverses the road graph – ensuring that even for imperfect road locations and positioning, the method stays reliable. Traffic flow simulations are primarily interested in traffic loads, road properties, and the way the roads are subdivided by intersections, among many other aspects. Mostly, they are not concerned with the geographical locations of the roads – as long as the lengths of the roads are correct, any geographical permutation will yield invariant results.

However, some applications are concerned with the road network in the context of its surroundings, which makes the accuracy of its georeferencing a relevant topic. Noise modelling is one such example, because the propagation of road noise is sensitive to the height of the road itself as well as to the immediate surroundings of the roads, requiring exact geographical locations of non-road objects surrounding the road (such as buildings or noise barriers), and the terrain. However, a 2D representation of the road network can only describe lateral positions, a factor that may limit its use in applications where elevation is important. For instance, consider two buildings at the side of a road, a tall one behind a shorter one. A 2D semantic road model would tell us that the entirety of the taller building receives a decreased noise load because the shorter one – standing between the taller one and the road – suppresses it. In reality, this observation is invalid because the part of the taller building which is visible from the road above the shorter building receives the full noise load. To be able to represent such 3D relationships in our model correctly, we need both buildings and roads to be three-dimensional.

2D-projected digital road models with mediocre accuracy have attracted great scientific and commercial attention in the past two decades (especially since the advent of digital cartography and satellite navigation). However, accurate 3D representations are still atypical, owing to factors such as increased cost of generation and maintenance, increased complexity of visualisation and analysis, and a lack of significant use cases (relative to the great number of uses cases for 2D road models). As a result, 2D road models are common both in terms of public and private geospatial providers, whereas accurate 3D road models are rare in comparison.

When a use case arises and an accurate 3D model is needed, the provider generally faces two choices: to produce and entirely new model, or to enrich its pre-existing 2D model with elevations. The decision generally depends on the quality of the pre-existing data set relative to the requirements for the 3D model, as well as that of the datasets available as sources of elevation with which the model can be enriched (as well as an other restrictions placed on the model specific to the use case). In terms of the source of elevations, the rule of thumb in the geospatial field is that data acquisition is vastly expensive relative to re-using pre-existing datasets, especially openly available ones. As a result, many providers first attempt to find a way to convert their datasets into three dimensions from existing data.

In certain projects the accuracy requirement and restrictions on the modelling procedure may be prescribed legally. Such is the case for the client of this proposed TU Delft MSc dissertation research, the \textit{Nationaal Dataportaal Wegverkeer} (NDW, National Road Traffic Data Portal), part of \textit{Rijkswaterstaat} (RWS, Directorate-General for Public Works and Water Management), a Dutch government organisation who are in the process of enriching their pre-existing open data 2D road model, called \textit{Nationaal Wegenbestand} (NWB, National Road Database) with 3D data, to attain compliance with \textit{SWUNG2}, the new version of the Dutch noise legislation or \textit{geluidwetgeving}. The new version prescribes, among other things, accuracy requirements for the road model underlying the noise simulations, with explicit mention of it having to be three-dimensional. Due to cost considerations and reasons related to NDW’s data acquisition pipeline, the pre-existing 2D realisation of NWB will be converted into a 3D dataset (dubbed \textit{3D-NWB}) primarily using open data geospatial datasets. They have produced a prototype realisation themselves, and subsequently contracted the consultant firm \textit{Royal HaskoningDHV} (RHDHV) to create a commercial implementation based on their experience with the prototype. The development of this tool was concluded in December 2020. In addition to simply using the commercial implementation, they wish to assess how it fares in terms of spatial accuracy-related SWUNG2-compliance. This dissertation research will attempt to contribute to this assessment by producing an original implementation that favours accuracy, and in which accuracy can be scientifically proven. The scientific implementation will serve as a reference to which its commercial counterpart can be compared to assess its accuracy. This document presents the preliminary findings and project plans of the proposed research. More detailed descriptions of the research questions and methodology will be presented in later sections.

In this project, I will primarily focus on a Lidar point cloud and a 3D topographical line dataset as elevation sources. These will be described in more depth in the Tools and datasets section. While in the topographical dataset care has been taken never to include objects that overlap in three dimensions, in the point cloud objects are often described that overlap significantly, when seen from above, such as ground surfaces covered in vegetation. The question arises, how I should deal with such real-world geometries, which in turn leads us to the discussion of which exact field of GeoScience my project is positioned in. The topic itself is strongly related to the GIS and geomatics fields. One of the reasons why 2D road models are popular, is that their geometry and network properties can be analysed using a multitude of well-proven GIS methods and software kits. In such models, if elevation measurements exist, they are only present as an \textit{elevation attribute} (i.e. a semantic data field, like street names), because the geometrical model does not support true 3D operations. This is conceptually identical to projecting the geometries onto the horizontal plane. Geometric models that treat the vertical dimension explicitly are more common in the geomatics field; namely 2.5D and 3D models. While using 2.5D models restricts the types of physical entities that can be modelled, it also greatly simplifies certain types of analysis conceptually and computationally, in turn making it ideal for working with data on similar scales to GIS (on a the national scale for instance, as in this research). Much like the concept of divide-and-conquer in computer science, the fundamental concept of applying 2.5D-based methods to 3D problems is to decompose them into smaller sub-problems until they become solvable using efficient 2.5D operations. Our research is positioned in the geomatics field because its implementation will not only be driven by accuracy; it will also focus on how 2.5D methods can be applied in a way that enables the processing of the tremendous point cloud size that occurs on a national level. The divide-and-conqer concept will be applied to decompose the road network into segments that can be locally regarded as \textit{terrain}, and hence be modelled in 2.5D. Considering the road network as a whole, we may observe that this assumption is violated occasionally (consider motorway interchanges for instance). Decomposition ensures that each road surface is considered individually, thus eliminating the need  As such, this proposed research overlaps with the geomatics field of digital terrain modelling, in terms of how it will generate and store the digital representation of the terrain, and as a consequence, the manner in which it will derive elevations from it. It also touches on the digital terrain modelling topic of handling a terrain representation that is too large to fit into memory at once. Furthermore, it overlaps with the geomatics field of feature extraction, which we take inspiration from when locating the representation of road features in the Lidar data.





