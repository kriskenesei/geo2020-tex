%!TEX root = ../thesis_proposal.tex

\chapter{Introduction}
\label{chap:i}

\textit{Section being written}

Constructing and maintaining an up-to-date graph-like road network on the national level has a range of firmly established uses. Owing to its structure, it can be used efficiently for modelling and simulation purposes, such as for traffic flow simulations, construction and re-construction impact modelling (to pinpoint optimal locations and types of investment), and noise propagation modelling (where the roads are a source of noise load). It can also be used for navigation; a graph-like road network representation is at the heart of all road navigation services. Combined with other data sets, we can mention an even wider range of use cases: complemented by ecological statistics and models, it can offer insight into the impact of roads, and future road construction on the ecological impact on the flora and fauna in the vicinity of the roads. Or to generalise, a good model representing the road network in a digital format may be used as a shared working space when aggregating geospatial data relating to roads from various sources. It makes it possible for geographical road locations, topographical relationships, and arbitrary semantic information to reside in the same data model, making analysis techniques more straightforward, enforcing consistency and saving effort for data providers who would otherwise all need to maintain their own road model.

One may remark that a two-dimensional representation with approximate geographical locations may suffice for many of these purposes, topology being the main concern in network analysis. For instance, GNSS navigation software often use snapping methods to ensure that the navigating vehicle always traverses the road graph – ensuring that even for imperfect road locations and positioning, the method stays reliable. Traffic flow simulations are primarily interested in traffic loads, road properties, and the way the roads are subdivided by intersections, among many other aspects. Mostly, they are not concerned with the geographical locations of the roads – as long as the lengths of the roads are correct, any geographical permutation will yield invariant results.

However, some applications are concerned with the road network in the context of its surroundings, which makes the accuracy of its georeferencing a relevant topic. Noise modelling is one such example, because the propagation of road noise is sensitive to the height of the road itself as well as to the immediate surroundings of the roads, requiring exact geographical locations of non-road objects surrounding the road (such as buildings or noise barriers), and the terrain. However, a 2D representation of the road network can only describe lateral positions, a factor that may limit its use in applications where elevation is important. For instance, consider two buildings at the side of a road, a tall one behind a shorter one. A 2D semantic road model would tell us that the entirety of the taller building receives a decreased noise load because the shorter one – standing between the taller one and the road – suppresses it. In reality, this observation is invalid because the part of the taller building which is visible from the road above the shorter building receives the full noise load. To be able to represent such 3D relationships in our model correctly, we need both buildings and roads to be three-dimensional.

2D-projected digital road models with mediocre accuracy have attracted great scientific and commercial attention in the past two decades (especially since the advent of digital cartography and satellite navigation). However, accurate 3D representations are still atypical, owing to factors such as increased cost of generation and maintenance, increased complexity of visualisation and analysis, and a lack of significant use cases (relative to the great number of uses cases for 2D road models). As a result, 2D road models are common both in terms of public and private geospatial providers, whereas accurate 3D road models are rare in comparison.

When a use case arises and an accurate 3D model is needed, the provider generally faces two choices: to produce and entirely new model, or to enrich its pre-existing 2D model with elevations. The decision generally depends on the quality of the pre-existing data set relative to the requirements for the 3D model, as well as that of the datasets available as sources of elevation with which the model can be enriched (as well as an other restrictions placed on the model specific to the use case). In terms of the source of elevations, the rule of thumb in the geospatial field is that data acquisition is vastly expensive relative to re-using pre-existing datasets, especially openly available ones. As a result, many providers first attempt to find a way to convert their datasets into three dimensions from existing data.

In certain projects the accuracy requirement and restrictions on the modelling procedure may be prescribed legally. Such is the case for the client of this MSc research, the Nationaal Dataportaal Wegverkeer (NDW, National Road Traffic Data Portal), part of Rijkswaterstaat (RWS, Directorate-General for Public Works and Water Management), a Dutch government organisation who are in the process of enriching their pre-existing open data 2D road model, called Nationaal Wegenbestand (NWB, National Road Database) with 3D data, to attain compliance with SWUNG2, the new version of the Dutch noise legislation (geluidwetgeving). The new version prescribes, among other things, accuracy requirements for the road model underlying the noise simulations, with explicit mention of it having to be three-dimensional. Due to cost considerations and reasons related to NDW’s data acquisition pipeline, the pre-existing 2D realisation of NWB will be converted into a 3D dataset (dubbed “3D-NWB”) primarily using open data geospatial datasets. They have produced a prototype realisation themselves, and subsequently contracted the consultant firm Royal HaskoningDHV (RHDHV) to create a commercial implementation based on their experience with the prototype. The development of this tool was concluded in December 2020. In addition to simply using the commercial implementation, they wish to analyse and assess how it fares in terms of spatial accuracy-related SWUNG2-compliance. This dissertation research will attempt to contribute to this analysis and assessment by performing an independent literature review and of how accuracy can be tracked throughout necessary processing pipeline, and will present the results of an original implementation in addition to the benchmark of its commercial counterpart. This document presents the preliminary findings and further project plans of the above TU Delft research, proposed as my graduation project. More detailed descriptions of the research questions and methodology will be presented in later sections.

