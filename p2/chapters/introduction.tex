%!TEX root = ../thesis_proposal.tex

\chapter{Introduction}
\label{chap:i}

\emph{[Section being written]}

Constructing and maintaining an up-to-date graph-like road network on the national level has a range of firmly established uses. Owing to its structure, it can be used efficiently for modelling and simulation purposes, such as for traffic flow simulations, construction and re-construction impact modelling (to pinpoint optimal locations and types of investment), and noise propagation modelling (where the roads are a source of noise load). It can also be used for navigation; a graph-like road network representation  is at the heart of all road navigation services. Combined with other data sets, we can mention an even wider range of use cases: complemented by ecological statistics and models, it can offer insight into the impact of roads, and future road construction on the ecological impact on the flora and fauna in the vicinity of the roads. Or to generalise, a good model representing the road network in a digital format may be used as a shared working space when aggregating geospatial data relating to roads from various sources. It makes it possible for geographical road locations, topographical relationships, and arbitrary semantic information to reside in the same data model, making analysis techniques more straightforward, enforcing consistency and saving effort for data providers who would otherwise all need to maintain their own road model.