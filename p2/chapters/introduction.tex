%!TEX root = ../thesis_proposal.tex

\section{Introduction}
\label{sec:i}

Constructing and maintaining an up-to-date graph-like road network on the national level has a range of firmly established uses. Owing to its structure, it can be used efficiently for modelling and simulation purposes, such as for traffic flow simulations, passenger transport modelling, construction and re-construction impact modelling (to pinpoint optimal locations and types of investment), and road noise load modelling (\cite{bell_iida_1997}, \cite{zhu_li_2007}, \cite{zhang_2011}, \cite{duran_santos_2014}, \cite{peng_etal_2020}). It can also be used for navigation; a graph-like road network representation is at the heart of all road navigation services (\cite{yue_etal_2008}). Combined with other data sets, we can mention an even wider range of use cases: complemented by ecological statistics and models, it can offer insight into the impact of roads, and future road construction on the ecological impact on the flora and fauna in the vicinity of the roads. Or to generalise, a good model representing the road network in a digital format may be used as a shared working space when aggregating geospatial data relating to roads from various sources. It makes it possible for geographical road locations, topographical relationships, and arbitrary semantic information to reside in the same data model, making analysis techniques more straightforward, enforcing consistency and saving effort for data providers who would otherwise all need to maintain their own road model (\cite{ekpenyong_etal_2007}).

One may remark that a two-dimensional representation with approximate geographical locations may suffice for many of these purposes, topology being the main concern in network analysis. For instance, GNSS navigation software often use snapping methods to ensure that the navigating vehicle always traverses the road graph – ensuring that even for imperfect road locations and positioning, the method stays reliable (\cite{chen_hsu_2020}). Traffic flow simulations are primarily interested in traffic loads, road properties, and the way the roads are subdivided by intersections, among many other aspects. Mostly, they are not concerned with the geographical locations of the roads – as long as the lengths of the roads are correct, any geographical permutation will yield invariant results (\cite{thomson_richardson_1995}).

However, some applications are concerned with the road network in the context of its surroundings, which makes the accuracy of its georeferencing a relevant topic. Noise modelling is one such example, because the propagation of road noise is sensitive to the height of the road itself as well as to the immediate surroundings of the roads, requiring exact geographical locations of non-road objects surrounding the road (such as buildings or noise barriers), and the terrain (\cite{ishiyama_etal_1991}, \cite{bennett_1997}, \cite{guarnaccia_quartieri_2012}). However, a 2D representation of the road network can only describe lateral positions, a factor that may limit its use in applications where elevation is important. For instance, consider two buildings at the side of a road, a tall one behind a shorter one. A 2D semantic road model would tell us that the entirety of the taller building receives a decreased noise load because the shorter one – standing between the taller one and the road – suppresses it. In reality, this observation is invalid because the part of the taller building which is visible from the road above the shorter building receives the full noise load. To be able to represent such 3D relationships in our model correctly, we need both buildings and roads to be three-dimensional.

2D-projected digital road models with mediocre accuracy have attracted great scientific and commercial attention since the advent of digital cartography and satellite navigation (\cite{taylor_etal_2001}, \cite{yue_etal_2008}, \cite{chen_hsu_2020}). However, \textit{accurate 3D representations} are still atypical, owing to factors such as increased cost of generation and maintenance, increased complexity of visualisation and analysis, and a lack of significant use cases (\cite{zhu_li_2007}, \cite{wang_etal_2014}). As a result, 2D road models are common both in terms of public and private geospatial providers, whereas accurate 3D road models are rare in comparison.

When a use case arises and an accurate 3D model is needed, the provider generally faces two choices: to produce a new model, or to enrich a pre-existing 2D model with elevation data. The decision generally depends on the quality of the available 2D data set relative to the requirements for the 3D model, as well as that of the dataset(s) available as sources of elevation with which the model can be enriched, among other reasons (\cite{zhu_li_2007}, \cite{zhu_li_2008}, \cite{wang_etal_2014}). In terms of the source of elevation data, the rule of thumb in the geospatial field is that data acquisition is vastly expensive relative to re-using existing datasets, especially openly available ones. As a result, many providers first attempt to find a way to convert their datasets into 3D using existing data.

In certain projects the accuracy requirement and restrictions on the modelling procedure may be prescribed legally. Such is the case for the client of this proposed TU Delft MSc dissertation research, the \textit{Nationaal Dataportaal Wegverkeer} (NDW, National Road Traffic Data Portal), part of \textit{Rijkswaterstaat} (RWS, Directorate-General for Public Works and Water Management), a Dutch government organisation who are in the process of enriching their pre-existing open data 2D road model, called \textit{Nationaal Wegenbestand} (NWB, National Road Database) with 3D data, to attain compliance with \textit{SWUNG2}, the new version of the Dutch noise legislation or \textit{geluidwetgeving}. The new version prescribes, among other things, accuracy requirements for the road model underlying the noise simulations, with explicit mention of it having to be three-dimensional. Due to cost considerations and reasons related to NDW’s data acquisition pipeline, the pre-existing 2D realisation of NWB will be converted into a 3D dataset (dubbed \textit{3D-NWB}) primarily using open data geospatial datasets. They have produced a prototype realisation themselves, and subsequently contracted the consultant firm \textit{Royal HaskoningDHV} (RHDHV) to create a commercial implementation based on their experience with the prototype. The development of this tool was concluded in December 2020. In addition to simply using the commercial implementation, they wish to assess how it fares in terms of spatial accuracy-related SWUNG2-compliance. This dissertation research will attempt to contribute to this assessment by producing an original implementation that favours accuracy, and in which accuracy can be \textit{tracked} and \textit{quantitatively evaluated}. It will also serve as a reference to which its commercial counterpart can be compared so that its accuracy can be evaluated indirectly. This document presents the preliminary findings and project plans of the proposed research.

I will primarily focus on a Lidar point cloud and a 3D topographical line dataset as elevation sources. In both of these datasets, it is evidenced that roads are in truly three-dimensional relationships with one another, and with their environment. For instance, they cross above and below other roads, and are often found underground, in tunnels. The question arises, how such real-world geometries should be dealt with, which is a question whose answer effectively defines which exact field of geoscience my project is positioned in. The two candidates in the context of road network modelling are GIS and geomatics. It is thus worth discussing briefly how each treats 3D objects typically. In terms of GIS, one of the reasons why 2D road models are popular is that their geometry and network properties can be analysed using a multitude of well-proven GIS methods and software kits. In GIS models, even if elevation measurements exist, they are only present as an \textit{elevation attribute} (i.e. a semantic data field, like street names), because GIS geometrical models do not typically support true 3D operations. This is conceptually identical to projecting the geometries onto the horizontal plane. Geometric models that treat the vertical dimension explicitly are more common in geomatics; namely 2.5D and 3D models. While using 2.5D models restricts the types of physical entities that can be modelled, it also greatly simplifies certain types of analysis conceptually and computationally. This makes it ideal for modelling on similar scales to GIS; on the national scale for instance, as in this research. While 2.5D modelling initially appears to be a good candidate for this project, it is generally unsuitable for the 3D relationships roads have with each other, and their environment. However, much like how the concept of divide-and-conquer works in computer science, it is possible decompose 3D problems into smaller sub-problems until they become natively compatible with 2.5D methods. This research is positioned in the geomatics field because my implementation will explore how 2.5D methods can be applied in a way that enables the \textit{piecewise} modelling of a national road network. The divide-and-conqer concept will be applied to decompose the road network into segments that can be individually locally regarded as \textit{terrain} and hence be modelled in 2.5D. As the proposed research focuses on 2.5D methods, it overlaps with the geomatics field of digital terrain modelling in terms of how it will generate and store the digital representation of road surfaces, and as a consequence, the manner in which it will derive elevations from them. As the Methodology section will reveal, it also strongly overlaps with the geomatics field of 3D feature extraction (and to a small extent, photogrammetry), because of how I propose to derive the 2.5D road surface models from our input datasets. As also indicated by the research questions, it is among the specific aims to study how a combination of mainly geomatics-based tools and methods can be used to accomplish the tasks required by NDW, as well as to assess their accuracy and suitability when used in this way.

Care will be taken to examine how the same challenges as we expect to face regarding output accuracy and 3D road interactions are handled by the commercial implementation. Our preliminary analysis of the implementation indicates for instance, that it uses input data with non-validated accuracy for motorways, and simple bilinear interpolation in badly interpolated, Lidar-based raster DTMs for provincial roads. Furthermore, it only handles 3D interactions explicitly for motorways and even for these, it does so via a series of assumptions that do not hold in general. This comparison may provide an insight into some aspects of 3D geospatial data processing that are frequently overlooked in commercial implementations.

To summarise, the proposed research is comprised of taking NWB, applying a pipeline of mostly geomatics operations, and outputting an an elevation-enriched version of it. In intermediate steps, it will be decomposed into subsets that are solvable by 2.5D methods, and reassembled into a 3D-enriched version of the input geometries. A byproduct of the procedure will be accurate piecewise 3D road surface models. The accuracy will be tracked throughout the procedure.