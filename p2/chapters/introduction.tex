%!TEX root = ../thesis_proposal.tex

\chapter{Introduction}
\label{chap:i}

\textit{Section being written}

Constructing and maintaining an up-to-date graph-like road network on the national level has a range of firmly established uses. Owing to its structure, it can be used efficiently for modelling and simulation purposes, such as for traffic flow simulations, construction and re-construction impact modelling (to pinpoint optimal locations and types of investment), and noise propagation modelling (where the roads are a source of noise load). It can also be used for navigation; a graph-like road network representation is at the heart of all road navigation services. Combined with other data sets, we can mention an even wider range of use cases: complemented by ecological statistics and models, it can offer insight into the impact of roads, and future road construction on the ecological impact on the flora and fauna in the vicinity of the roads. Or to generalise, a good model representing the road network in a digital format may be used as a shared working space when aggregating geospatial data relating to roads from various sources. It makes it possible for geographical road locations, topographical relationships, and arbitrary semantic information to reside in the same data model, making analysis techniques more straightforward, enforcing consistency and saving effort for data providers who would otherwise all need to maintain their own road model.

One may remark that a two-dimensional representation with approximate geographical locations may suffice for many of these purposes, topology being the main concern in network analysis. For instance, GNSS navigation software often use snapping methods to ensure that the navigating vehicle always traverses the road graph – ensuring that even for imperfect road locations and positioning, the method stays reliable. Traffic flow simulations are primarily interested in traffic loads, road properties, and the way the roads are subdivided by intersections, among many other aspects. Mostly, they are not concerned with the geographical locations of the roads – as long as the lengths of the roads are correct, any geographical permutation will yield invariant results.

However, some applications are concerned with the road network in the context of its surroundings, which makes the accuracy of its georeferencing a relevant topic. Noise modelling is one such example, because the propagation of road noise is sensitive to the height of the road itself as well as to the immediate surroundings of the roads, requiring exact geographical locations of non-road objects surrounding the road (such as buildings or noise barriers), and the terrain. However, a 2D representation of the road network can only describe lateral positions, a factor that may limit its use in applications where elevation is important. For instance, consider two buildings at the side of a road, a tall one behind a shorter one. A 2D semantic road model would tell us that the entirety of the taller building receives a decreased noise load because the shorter one – standing between the taller one and the road – suppresses it. In reality, this observation is invalid because the part of the taller building which is visible from the road above the shorter building receives the full noise load. To be able to represent this, we need a 3D model of the vicinity of the roads, which in turn means that to ensure that the results are accurate, we also need a 3D road model; projecting the roads onto the horizontal plane would yield inaccurate noise source locations. For instance, elevated roads and roads traversing hills would be represented incorrectly. In summary, the significance of the third dimension depends on the specific application. 