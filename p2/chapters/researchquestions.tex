%!TEX root = thesis_proposal.tex

\chapter{Research questions}
\label{chap:rq}

The decision was made to include the research questions before the Related work section. The justification for this change is that in this particular research, the research questions primarily emerged from the requirements of the NDW assignment and the suspected shortcomings of the commercial implementation. The literature review was \textit{targeted} at identifying a range of methods that could be used to complete the 3D enrichment of NDW while ensuring and quantifying output accuracy and resolving the suspected issues with the commercial implementation. The research questions are presented as a range of primary questions along with secondary sub-questions.


\begin{enumerate}
\item Find a combination of geomatics methods that can perform the elevation-enrichment of NWB in a similar manner to the commercial implementation, but eliminating suspected shortcomings as best as possible.
\begin{enumerate}
    \item Attempt to use the same datasets as the commercial implementation.
    \item Find and implement a workflow to decompose intrinsically 3D problem into a collection of smaller problems that can be solved by 2.5D methods.
    \item Solve the scaling-related issues optimally by building on the same decomposition as the one needed to ensure 2.5D-compliance.
    \item Building once again on the decomposition, ensure that efficient updates are possible as new data arrives or old data is updated.
    \item Benchmark the solution in areas of complex 3D relationships and optimise the implementation accordingly.
    \item Benchmark the solution in areas where input data is scarce due to occlusion or other reasons and optimise the implementation accordingly.
    \item Explore whether lines representing the vicinity of the roads can be enriched with elevations in the same way.
\end{enumerate}
\item Examine by experimentation and analysis, how Dutch open data geospatial datasets can be used to enrich NWB with elevations in a manner that can conform with external spatial accuracy requirements.
\begin{enumerate}
\item Research and assess the initial accuracy of the input datasets. This includes NWB itself.
\item Track changes in accuracy throughout the procedure
\end{enumerate}
\end{enumerate}