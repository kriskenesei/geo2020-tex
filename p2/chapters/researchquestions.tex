%!TEX root = thesis_proposal.tex

\section{Research questions}
\label{sec:rq}

The decision was made to include the research questions before the Related work section (\ref{sec:rw}). The justification for this change is that in this particular research, the research questions primarily emerged from the requirements of the NDW assignment and the suspected shortcomings of the commercial implementation. The literature review was \textit{targeted} at identifying a range of methods that could be used to complete the 3D enrichment of NDW while ensuring and quantifying output accuracy and resolving the suspected issues with the commercial implementation, as well as our own questions regarding the planned 2.5D-based methodology. The research questions are thus arranged into two main groups: practical questions relating to building a mainly 2.5D methodology, and accuracy-related questions.

\begin{enumerate}
    \item Is it possible to perform the elevation-enrichment of NWB in a similar manner to the commercial implementation, but eliminating suspected shortcomings as best as possible, and relying on an efficient, predominantly 2.5D-based geomatics methodology?
    \begin{enumerate}
        \item What is the exact methodology of the commercial implementation? What do we suspect the theoretical shortcomings to be based on RHDHV's methodology and output?
        \item Can my methodology by built around the same datasets as the ones used in the commercial implementation? What are the characteristics of these datasets?
        \item What types of research has been carried out in this specific field? What can we learn from existing research? How successful were they?
        \item Is it possible to base the workflow entirely on 2.5D methods by decomposing this intrinsically 3D problem into a collection of smaller problems?
        \item Can such a method of decomposition be used to simultaneously solve the scaling issues related to handling a national Lidar point cloud input dataset?
        \item Is it possible to build the workflow in a way that it allows efficient update operations to be carried out as new data arrives or old data is updated?
        \item How do temporal inconsistencies manifest themselves in the procedure, and in the output? What problems may arise, and how best to solve them?
        \item How well does the implementation perform in areas of complex 3D road relationships? How can this performance be assessed visually and quantitatively using real-world examples, such as multi-level motorway exchanges and roads on long, elevated civil engineering structures?
        \item How well does the implementation perform in areas where input elevation data is scarce, or missing? How can this performance best be assessed visually and quantitatively using real-world examples, such as dense vegetation cover or other objects occluding roads and roads constructed in tunnels?
        \item How can we best assess the overall performance of the implementation against the commercial implementation in key locations such as the above?
        \item How smooth is the output? Are sudden jumps introduced after aggregating the decomposed model? If so, can this be resolved by optimising the procedure, or are additional smoothing steps necessary?
        \item Can the same workflow be used to also derive elevations for lines that a fixed distance away from the NWB centrelines, representing the \textit{vicinity} of roads?
        \item Can the workflow be used to optimise the \textit{horizontal} location of NWB centrelines?
        \item Can the workflow serve as an aggregator of elevation data from small scale sources such as road management datasets?
        \item The workflow is planned to produce surface models of \textit{road segments}. What would be needed to aggregate these into a global model containing all roads?
    \end{enumerate}
    \item Can the workflow be developed in a way that it can guarantee compliance with prescribed elevation accuracy?
    \begin{enumerate}
        \item What types of accuracy may we distinguish between? Topological accuracy? Lateral accuracy? Purely elevation accuracy? How does each type of uncertainty affect the output?
        \item In pre-existing research, what methods have been used to measure accuracy and estimate the influence of processing steps on accuracy?
        \item What is the accuracy of our elevation sources? Can we trust elevation data sources with undocumented accuracy?
        \item What is the effect of uncertainty in the lateral positions of NWB centrelines? Could the impact of this be reduced by optimising the lateral location of NWB?
        \item What do we need to be able to track the evolution of the accuracy throughout the workflow? Can we build it exclusively from steps that prevent serious degradation to the accuracy, and also make any degradation quantifiable?
        \item Is uncertainty also influenced by geographically variable factors? If so, how can we quantify these and include them in the output estimation of uncertainty? How can we minimise their effects?
        \item How does the addition of small-scale elevation data sources influence accuracy? How can they be best made use of in this context?
        \item Would it be possible to develop the solution in a way that it can indicate where levels of uncertainty drop below the prescribed threshold, and why?
        \item If smoothing steps are necessary as a post-processing step, how can we be certain that the vertical displacements do not corrupt accuracy? Are topological constraints necessary? If so, how can they be enforced?
        \item How may we measure or estimate the accuracy of the commercial implementation? Can it be tracked throughout their procedure starting from input accuracy, or will we need to rely on more indirect methods?
    \end{enumerate}
\end{enumerate}