%!TEX root = thesis_proposal.tex

\chapter{Related work}
\label{chap:rw}

Many relevant research papers have been located and examined as part of the literature review stage of this project. There are two main areas that are of interest in the context of this research, which follows from the research questions. The literature review concerning thes two areas will be presented below in two separate sections.

\section*{Accuracy description of Lidar sensing and Lidar-based DEM-generation}

By Lidar-based DEM generation we mean any 2.5D model that can be used to represent the terrain digitally, in particular TIN-based methods which we suspect to be well suited to this project owing to their ability to represent detail at spatially inhomogeneous levels. Furthermore, our second area of interest is feature extraction because our intention is to not only systematically query the DEM for elevations around the NWB road centrelines, but to take into account only road points around them, with as best a level of completeness as possiblethe entire road surface as represented by the point cloud. This requires one to identify, or at least approximate the representation of the roadelevation points inside the point cloudroad. It is worth noting that ourThis review focused on Lidar-based elevation measurement. The reason for this is that we consider AHN our first priority in terms of source of measurements, given the lack of information about the accuracy and exact acquisition methods relevant to DTB, as well as the relatively straightforward nature of extracting road elevations from it (as demonstrated by the commercial implementation). Our proposed workflow was distilled mostly from what is described in this this section and is presented in the Methodology section.